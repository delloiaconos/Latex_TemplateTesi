\usepackage{setspace}
\usepackage[gen]{eurosym}
\usepackage{rotating} % sidewaytables
\usepackage{graphicx} % Required for inserting images
\usepackage{geometry}           % impostazione generale di pagina
\usepackage{emptypage}          % per lasciare bianche le pagine senza testo
\usepackage[babel]{csquotes}    % per la bibliografia
\usepackage{biblatex}           % per la bibliografia
\usepackage{lipsum}             % inserire lorem ipsum \lipsum[]
\usepackage{caption}            % per le didascalie
\usepackage{subcaption}

\usepackage{booktabs}           % per le tabelle
\usepackage{pdfpages}           % inserire pdf esterni

\usepackage{siunitx}            % unità di misura del SI
\sisetup{separate-uncertainty}

\usepackage{hyperref}           % collegamenti ipertestuali e segnalibri
\hypersetup{hidelinks}
\urlstyle{same}


%% COMMAND: figref adn tabref
\newcommand{\figref}[1]{Figura~\ref{#1}}
\newcommand{\tabref}[1]{Tabella~\ref{#1}}

\usepackage{xcolor}
%New colors defined below
\definecolor{codegreen}{rgb}{0,0.6,0}
\definecolor{codegray}{rgb}{0.5,0.5,0.5}
\definecolor{codepurple}{rgb}{0.58,0,0.82}
\definecolor{backcolour}{rgb}{0.95,0.95,0.92}
\definecolor{codeblack}{rgb}{0.95,0.95,0.92}
\definecolor{codemauve}{rgb}{0.58,0,0.82}

%\usepackage[colorlinks=true, allcolors=blue]{hyperref} % link
% codice programmi 
\usepackage{listings}
\lstset{ 
  backgroundcolor=\color{backcolour},   % choose the background color; you must add \usepackage{color} or \usepackage{xcolor}; should come as last argument
  basicstyle=\fontsize{10}{12}\ttfamily,
  %basicstyle=\footnotesize\ttfamily,        % the size of the fonts that are used for the code
  breakatwhitespace=false,         % sets if automatic breaks should only happen at whitespace
  breaklines=true,                 % sets automatic line breaking
  captionpos=b,                    % sets the caption-position to bottom
  commentstyle=\color{codegreen},    % comment style
  deletekeywords={...},            % if you want to delete keywords from the given language
  escapeinside={\%*}{*)},          % if you want to add LaTeX within your code
  extendedchars=true,              % lets you use non-ASCII characters; for 8-bits encodings only, does not work with UTF-8
  frame=single,	                   % adds a frame around the code
  keepspaces=true,                 % keeps spaces in text, useful for keeping indentation of code (possibly needs columns=flexible)
  keywordstyle=\color{codepurple},       % keyword style
  language=C,                 % the language of the code
  morekeywords={*,...},            % if you want to add more keywords to the set
  numbers=none,                    % where to put the line-numbers; possible values are (none, left, right)
  numbersep=5pt,                   % how far the line-numbers are from the code
  numberstyle=\tiny\color{codegray}, % the style that is used for the line-numbers
  rulecolor=\color{black},         % if not set, the frame-color may be changed on line-breaks within not-black text (e.g. comments (green here))
  showspaces=false,                % show spaces everywhere adding particular underscores; it overrides 'showstringspaces'
  showstringspaces=false,          % underline spaces within strings only
  showtabs=false,                  % show tabs within strings adding particular underscores
  stepnumber=5,                    % the step between two line-numbers. If it's 1, each line will be numbered
  stringstyle=\color{codemauve},     % string literal style
  tabsize=2,	                   % sets default tabsize to 2 spaces
  title=\lstname                   % show the filename of files included with \lstinputlisting; also try caption instead of title
}
\renewcommand\lstlistingname{Listato}
\renewcommand\lstlistlistingname{Listati}

\lstdefinestyle{codeAppendix}{
 basicstyle=\tiny\ttfamily\color{black}, % Colore del testo
  keywordstyle=\color{black}, % Colore delle parole chiave
  commentstyle=\color{black}, % Colore dei commenti
  stringstyle=\color{black}, % Colore delle stringhe
  numbers=left,
  numberstyle=\tiny\color{black}, % Colore dei numeri di riga
  breaklines=true,
  breakatwhitespace=true,
  showspaces=false,
  showstringspaces=false,
  showtabs=false,
  frame=none, % Nessuna cornice
  backgroundcolor=\color{white}, % Nessuno sfondo
  captionpos=b,
  tabsize=2
}

%% SDI: Pacchetti utilizzati per commenti e template.
%% A tesi conclusa dovrebbero poter essere eliminati.
\usepackage{comment}
\usepackage{lipsum}
\newenvironment{todo}{\color[rgb]{1.0,0.0,0.0}\noindent}{\newline}